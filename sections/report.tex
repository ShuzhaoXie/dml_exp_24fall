\chapter{实验报告撰写要求}
在撰写每项实验报告时,请注意并且避免出现以下问题:
\begin{itemize}
    \item 结构不完整。报告一般应包含以下几个关键要素:
    \begin{enumerate}
    \item \textbf{实验原理}:简单交代所要求实现算法的原理,向读者交代自己的理解,让批阅人快速批阅作者的理解是否正确。
    \item \textbf{代码实现}:通过截图或markdown等自带的代码块贴上关键性代码。
    \item \textbf{实验步骤,或运行参数、环境}:交代如何实现多节点、关键参数设置为何(简要交代即可,譬如batchsize、停止条件、训练数据集等)。
    \item \textbf{实验结果}:通过图片、表格等形式清晰地展示结果。
    \item \textbf{结果分析或结论}
    \end{enumerate}

    \item 截图影响可读性。
    \begin{itemize}
        \item 对代码的截图
        \begin{enumerate}
            \item 截图后字号过大、字号过小或相邻截图字号相差过大。
            \item 截图中,文本编辑器为深色背景,连续多块截图面积宽度不一致,导致看上去像一块块膏药,阅读起来非常难受。
        \end{enumerate}
        \item 对实验结果的截图
        \begin{enumerate}
            \item 终端的文字性的输出结果: 如果认为有必要截图证明自己完成了实验,可以作为中间结果,但仅作为最终结果输出会严重影响可读性。结果中关于最后几轮的loss,可以用曲线展示;结果中运行时间、准确性等可以通过表格展示。
            \item 曲线展示的结果: 需要对比时,最好把两条(多条)曲线放到一张图上对比展示,而不是两张图每张图上仅一条曲线。
        \end{enumerate}
    \end{itemize}

    \item 格式不清晰影响可读性。全部左对齐时,建议用markdown类似的格式。注意缩进、字体、字号变换、章节编号等。

    \item 交代不清晰。
    \begin{enumerate}
        \item 缺乏文字性阐述。对实验原理、实验步骤、实验结果等缺乏文字性阐述。对于新引入的符号缺乏说明,对多条曲线中的图例缺乏解释等。
        \item 缺乏单位。交代运行时间时缺乏单位等。
    \end{enumerate}
\end{itemize}

% ### 小实验报告出现的问题:
% 1. 结构不完整。报告一般应包含以下几个关键要素:
%     1. **实验原理**:简单交代所要求实现算法的原理,向读者交代自己的理解,让批阅人快速批阅作者的理解是否正确
%     2. **代码实现**:通过截图或markdown等自带的代码块贴上关键性代码。
%     3. **实验步骤,或运行参数、环境**:交代如何实现多节点、关键参数设置为何(简要交代即可,譬如batchsize、停止条件、训练数据集等)
%     4. **实验结果**:通过图片、表格等形式清晰地展示结果。
%     5. **结果分析或结论。**
% 1. 截图影响可读性。
%     1. **对代码的截图**:
%         1. 截图后字号过大、字号过小或相邻截图字号相差过大。
%         2. 截图中,文本编辑器为深色背景,连续多块截图面积宽度不一致,导致看上去像一块块膏药,阅读起来非常难受。
%     2. **对实验结果的截图**:
%         1. **终端的文字性的输出结果**,如果认为有必要截图证明自己完成了实验,可以作为中间结果,但仅作为最终结果输出会严重影响可读性。结果中关于最后几轮的loss,可以用曲线展示;结果中运行时间、准确性等可以通过表格展示。
%         2. **曲线展示的结果**,需要对比时,最好把两条(多条)曲线放到一张图上对比展示,而不是两张图每张图上仅一条曲线。
% 1. 格式不清晰影响可读性。全部左对齐时,用markdown类似的格式就很舒服了。以下不一定每一项都需要,但如果缺两项一般就让我不想看了。
%     1. **缺乏缩进**
%     2. **缺乏字体、字号变换**
%     3. **缺乏章节编号**
% 1. 交代不清晰。
%     1. **缺乏文字性阐述**。对实验原理、实验步骤、实验结果等缺乏文字性阐述。对于新引入的符号缺乏说明,对多条曲线中的图例缺乏解释等。
%     2. **缺乏单位**。交代运行时间时缺乏单位等。 -->